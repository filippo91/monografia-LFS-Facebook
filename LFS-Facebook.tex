\chapter{LFS-Facebook}

\section{Django Framework}
Django è un framework di alto livello per Python, che ha l'obiettivo di rendere facile e veloce lo sviluppo di applicazioni web.
.... forse spiego grossolanemente cosa fanna le parti che ho modificato in django....
 
\section{Progetti di partenza}
Una fase preliminare del tirocinio è stata impiegata principalmente nello studio di due progetti open source: LFS e django-facebook.
LFS è un negozio online che  si basa su Python, Django e jQuery. A questo è stato aggiunto poi un tema responsivo, ed ha rappresentato il progetto base sul quale si è lavorato per aggiungere delle funzionalità legate alla sua integrazione come applicazione fruibile su Facebook. 

Django-facebook, invece, permette ad applicazioni django di registrare gli utenti tramite backend Facebook, convertendo i dati dell'utente in un modello consono a django.

\section{Gestione Backend}
Un'esigenza che un progetto di un'applicazione potrebbe avere è quella di dover/voler pescare da un'altra sorgente di usernames e passwords.

Questo è proprio il nostro caso, vogliamo usare come backend di autenticazione quello di Facebook.
Per iniziare a prender un po' la mano abbiamo iniziato provando a far funzionare un'applicazione django come LFS con un LDAP. 

LDAP è un protocollo di livello applicativo per accedere e gestire servizi di directory su una rete IP. Uno degli usi più comuni di LDAP è quello di fornire un “single sign-on” dove per la coppia d'informazioni utente:password è condivisa tra diversi servizi.
Nel nostro caso è stato utilizzato openLDAP, un'implementazione opensource del suddetto protocollo.


Per aggiungere questa funzionalità ad un'applicazione Django è necessario modificare la variabile AUTHENTICATION BACKENDS del file setting.py, in particolare Django mantiene una lista di “authentication backends” che egli verifica per ogni autenticazione. 

%\begin{lstlisting}[frame=single]
%	AUTHENTICATION_BACKENDS = (
%    'django_auth_ldap.backend.LDAPBackend',
%    'django.contrib.auth.backends.ModelBackend',
%	)
%\end{lstlisting}

Quando qualcuno cerca di autenticarsi, Django prova l'autenticazione lungo tutta la lista di backend per cui è impostato. 
Attenzione che l'ordine è importante in quanto django si fermerà al primo match positivo.

%Con LDAP sono necessarie una serie di altre opzioni da aggiungere al file di setting, mentre per accedere al backend di autenticazione di Facebook è sufficente aggiungere in testa alla lista AUTHENTICATION_BACKENDS 'django_facebook.auth_backends.FacebookBackend'.

\section{Integrare un applicazione come Facebook App}
Per inserire l'applicazione tra le App di Facebook è necessario procurarsi innanzitutto le credenziali, in modo da.....
Qui i passi utili per ottenerle:
\begin{itemize}
	\item raggiungere la Facebook Developers Apps page;
	\item registrarsi come sviluppatore se non lo si è già;
	\item creare una nuova applicazione cliccando sull'aposito pulsante;
	\item accettare le normative della piattaforma Facebook;
\end{itemize}
Così facendo, sulla pagina principale dell'applicazione sono disponibili le due credenziali utili per il proseguo: ID applicazione e App Secret.
Adesso vanno inserite nel file di setting del progetto:

\begin{lstlisting}[frame=single]
FACEBOOK_APP_ID = 'YOUR APP ID'
FACEBOOK_APP_SECRET = 'YOUR APP SECRET NUMBER'
\end{lstlisting}
%FACEBOOK_CANVAS_PAGE = 'https://apps.facebook.com/%s/' % FACEBOOK_APP_ID
Da notare che è richiesta una connessione https.

\section{Protezione delle viste con autenticazione richiesta}
L'applicazione si presenta senza un link ad una pagina di login, in quanto l'utente sarà automaticamente rediretto alla procedura di autenticazione quando e se necessario. 

Per ottenere questo comportamento si è scelto di creare un decoratore che protegga le viste desiderate.
Le viste da proteggere non sono imposte dal sistema, ma è tutto pienamente configurabile attravrso il file di setting. Infatti, è stato creato un dizionario di variabili booleane, a ciascuna delle quali corrisponde una pagina che è possibile proteggere settando come 'True' la variabile in questione.

\begin{lstlisting}[frame=single]
VIEW_WITH_LOGIN_REQUIRED = {
    'add-to-cart': True, #True or False
    'shop': False, #True or False
    'category': False, #True or False
    'product': False, #True or False
    'checkout': False, #True or False
}
\end{lstlisting}

Se una vista è protetta da decoratore, sarà dunque scandito l'elenco di variabili: se la variabile corrispondente è uguale a True, sarà chiamato un altro decoratore, nativo del framework Django, @login\textunderscore required, che verificherà lo stato dell'utente, che se loggato procederà verso la pagina richiesta, altrimenti sarà rediretto alla pagina di login.

\begin{lstlisting}[frame=single]
def permissions_required(view_name):
    
    def decorator(func):
        if settings.VIEW_WITH_LOGIN_REQUIRED[view_name]:
            return login_required(func)
        else:
            return func

    return decorator
\end{lstlisting}

Comunque si può scegliere di non proteggere alcuna vista e permettere la libera navigazione all'utente che agirà come utente anonimo, potendo aggiungere prodotti al carrello, andare alla cassa, decidere di completare l'acquisto e quindi effettuare il pagamento.

\subsection{Pagina di login}
La pagina di login è stata modificata facendo un override di quella esistente attraverso il linguaggio di template di django, che permette, tra le altre cose, di ereditare da una pagina html e sovrascrivere i blocchi che si vuole.

\begin{lstlisting}[language=html]
 -> inheritance
 -> blocco da sovrascrivere
...

\end{lstlisting}

richiesta di connessione in poll 

\begin{lstlisting}[language=JavaScript]

var myVar = setInterval(function(){
    if(F !== undefined){
        F.connect(document.getElementById('facebook_form'), 
        ['email', 'user_about_me', 'user_birthday', 'user_website', 'user_likes']);
        clearInterval(myVar);
    }
    }, 500);

\end{lstlisting}

\subparagraph{template override}
L'override di un template pre-esistente viene effettuato creando un file html con lo stesso identico nome e dovrà essere riprodotta l'alberatura. Per esempio se voglio sovrascrivere lfs/templates/lfs/base.html, dovrò creare un file base.html in myapp/templates/lfs/base.html. 

Dopodichè è necessario anche intervenire nel file di setting ed assicurarsi che fra le INSTALLED\_ APPS, myapp appaia prima della applicazione di cui vogliamo sovrascrivere il template.

\subsection{Pagina di checkout}
\section{Personalizzazione del modello utente}
\section{Facebook plugin e API}
\section{Prodotti riservati ai fan}
\section{Internazionalizzazione}
\endinput