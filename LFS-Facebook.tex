\chapter{LFS-Facebook}

\section{Django Framework}
Django è un framework di alto livello per Python, che ha l'obiettivo di rendere facile e veloce lo sviluppo di applicazioni web.
.... forse spiego grossolanemente cosa fanna le parti che ho modificato in django....
 
\section{Progetti di partenza}
Una fase preliminare del tirocinio è stata impiegata principalmente nello studio di due progetti open source: LFS e django-facebook.
LFS è un negozio online che  si basa su Python, Django e jQuery. A questo è stato aggiunto poi un tema responsivo, ed ha rappresentato il progetto base sul quale si è lavorato per aggiungere delle funzionalità legate alla sua integrazione come applicazione fruibile su Facebook. 

Django-facebook, invece, permette ad applicazioni django di registrare gli utenti tramite backend Facebook, convertendo i dati dell'utente in un modello consono a django.

\section{Gestione Backend}
Un'esigenza che un progetto di un'applicazione potrebbe avere è quella di dover/voler pescare da un'altra sorgente di usernames e passwords.

Questo è proprio il nostro caso, vogliamo usare come backend di autenticazione quello di Facebook.
Per iniziare a prender un po' la mano abbiamo iniziato provando a far funzionare un'applicazione django come LFS con un LDAP. 

LDAP è un protocollo di livello applicativo per accedere e gestire servizi di directory su una rete IP. Uno degli usi più comuni di LDAP è quello di fornire un “single sign-on” dove per la coppia d'informazioni utente:password è condivisa tra diversi servizi.
Nel nostro caso è stato utilizzato openLDAP, un'implementazione opensource del suddetto protocollo.


Per aggiungere questa funzionalità ad un'applicazione Django è necessario modificare la variabile AUTHENTICATION BACKENDS del file setting.py, in particolare Django mantiene una lista di “authentication backends” che egli verifica per ogni autenticazione. 

%\begin{lstlisting}[frame=single]
%	AUTHENTICATION_BACKENDS = (
%    'django_auth_ldap.backend.LDAPBackend',
%    'django.contrib.auth.backends.ModelBackend',
%	)
%\end{lstlisting}

Quando qualcuno cerca di autenticarsi, Django prova l'autenticazione lungo tutta la lista di backend per cui è impostato. 
Attenzione che l'ordine è importante in quanto django si fermerà al primo match positivo.

%Con LDAP sono necessarie una serie di altre opzioni da aggiungere al file di setting, mentre per accedere al backend di autenticazione di Facebook è sufficente aggiungere in testa alla lista AUTHENTICATION_BACKENDS 'django_facebook.auth_backends.FacebookBackend'.

\section{Integrare un'applicazione come Facebook App}
Per inserire l'applicazione tra le App di Facebook è necessario procurarsi innanzitutto le credenziali, in modo da.....
Qui i passi utili per ottenerle:
\begin{itemize}
	\item raggiungere la Facebook Developers Apps page;
	\item registrarsi come sviluppatore se non lo si è già;
	\item creare una nuova applicazione cliccando sull'aposito pulsante;
	\item accettare le normative della piattaforma Facebook;
\end{itemize}
Così facendo, sulla pagina principale dell'applicazione sono disponibili le due credenziali utili per il proseguo: ID applicazione e App Secret.
Adesso vanno inserite nel file di setting del progetto:

\begin{lstlisting}[frame=single]
FACEBOOK_APP_ID = 'YOUR APP ID'
FACEBOOK_APP_SECRET = 'YOUR APP SECRET NUMBER'
\end{lstlisting}
%FACEBOOK_CANVAS_PAGE = 'https://apps.facebook.com/%s/' % FACEBOOK_APP_ID
Da notare che è richiesta una connessione https.

\section{Protezione delle viste con autenticazione richiesta}
L'applicazione si presenta senza un link ad una pagina di login, in quanto l'utente sarà automaticamente rediretto alla procedura di autenticazione quando e se necessario. 

Per ottenere questo comportamento si è scelto di creare un decoratore che protegga le viste desiderate.
Le viste da proteggere non sono imposte dal sistema, ma è tutto pienamente configurabile attravrso il file di setting. Infatti, è stato creato un dizionario di variabili booleane, a ciascuna delle quali corrisponde una pagina che è possibile proteggere settando come 'True' la variabile in questione.

\begin{lstlisting}[frame=single]
VIEW_WITH_LOGIN_REQUIRED = {
    'add-to-cart': True, #True or False
    'shop': False, #True or False
    'category': False, #True or False
    'product': False, #True or False
    'checkout': False, #True or False
}
\end{lstlisting}

Se una vista è protetta da decoratore, sarà dunque scandito l'elenco di variabili: se la variabile corrispondente è uguale a True, sarà chiamato un altro decoratore, nativo del framework Django, @login\textunderscore required, che verificherà lo stato dell'utente, che se loggato procederà verso la pagina richiesta, altrimenti sarà rediretto alla pagina di login.

\begin{lstlisting}[frame=single]
def permissions_required(view_name):
    
    def decorator(func):
        if settings.VIEW_WITH_LOGIN_REQUIRED[view_name]:
            return login_required(func)
        else:
            return func

    return decorator
\end{lstlisting}

Comunque si può scegliere di non proteggere alcuna vista e permettere la libera navigazione all'utente che agirà come utente anonimo, potendo aggiungere prodotti al carrello, andare alla cassa, decidere di completare l'acquisto e quindi effettuare il pagamento.

\subsection{Pagina di login}
La pagina di login è stata modificata facendo un override di quella esistente attraverso il linguaggio di template di django, che permette, tra le altre cose, di ereditare da una pagina html e sovrascrivere i blocchi che si vuole.

\begin{lstlisting}[language=html]
 -> inheritance
 -> blocco da sovrascrivere
...

\end{lstlisting}

richiesta di connessione in poll 

\begin{lstlisting}[language=JavaScript]

var myVar = setInterval(function(){
    if(F !== undefined){
        F.connect(document.getElementById('facebook_form'), 
        ['email', 'user_about_me', 'user_birthday', 
        'user_website', 'user_likes']);
        clearInterval(myVar);
    }
    }, 500);

\end{lstlisting}

\subparagraph{template override}
L'override di un template pre-esistente viene effettuato creando un file html con lo stesso identico nome e dovrà essere riprodotta l'alberatura. Per esempio se voglio sovrascrivere lfs/templates/lfs/base.html, dovrò creare un file base.html in myapp/templates/lfs/base.html. 

Dopodichè è necessario anche intervenire nel file di setting ed assicurarsi che fra le INSTALLED\_ APPS, myapp appaia prima della applicazione di cui vogliamo sovrascrivere il template.

\subsection{Pagina di checkout}
\section{Personalizzazione del modello utente}
Il modello è l'unica fonte di informazione a riguardo dei dati che vogliamo memorizzare. Esso contiene i campi e il comportamento per la memorizzazione. 

Quello che vogliamo fare è modificare il modello che definisce la gestione delle informazioni riferite all'utente. Per far questo, si è creato un file models.py, contenente una nuova classe che rappresenta il modello. Questa eredita da FacebookProfileModel la quale è la classe del pacchetto django-facebook che gestiva le informazioni ottenute dal backend di Facebook.

\begin{lstlisting}
class LfsFbUser(FacebookProfileModel):
    user = models.OneToOneField(User)
    fb_like_flag = models.BooleanField(default=False)
	
	...	
	
    class Meta():
            db_table = 'fb_data'
\end{lstlisting}

Per settare django affinchè usi il modello appena creato, bisogna modificare il file di setting, aggiungendo infondo al file oppure modificando la voce esistente, in modo da avere:
\begin{lstlisting}
AUTH_PROFILE_MODULE = 'lfs_facebook.LfsFbUser'
\end{lstlisting}


\section{Facebook plugin e API}
All'interno delle pagine html si è fatto uso di svariati plugin il codice è fornito in automatica agli sviluppatori registrati sulla piattaforma.

Inoltre si sono acquisite delle informazioni del profilo Facebook attraverso delle API che permettono di fare delle chiamate al database attraverso istruzioni FQL.

\section{Prodotti riservati ai fan}

%in catalog/patches.py

\begin{lstlisting}
fb_reserved = "False"
    for p in product.get_properties():
        if p.title == settings.FACEBOOK_FAN_RESERVED_PROPERTY:
            fb_reserved = "True"
\end{lstlisting}

%templates/lfs/catalog/product_base.html

\begin{lstlisting}
var fql_query = "SELECT uid 
				FROM page_fan 
				WHERE page_id = "+fbPage+
				" and uid = "+user_id;
    
    $(document).ready(FB.api({
        method: 'fql.query',
        query: fql_query
    },
    function(response){
        //FB.api('/me/permissions', function(response) {
        //});
        if(fbReserved == "True"){
            //product available facebook fan only,
            //check if user likes your facebook page
            if (response.length == 1 && response[0].uid == user_id){
                //he like it, so you able add-to-cart button
                //and show some message in the page
                allowPurchase(true);
            } else {
                //he doesn't like it, so you disable add-to-cart button
                //and show some message in the page
                allowPurchase(false);
            }
        }else{
            //product available for all
            $("#add-to-cart").prop('disabled', false);

            $("#fb-no-like").hide();
            $("#fb-already-like").hide();
        }
    }));
\end{lstlisting}

\section{Internazionalizzazione}
Lo sviluppo di un’applicazione web richiede, sempre più frequentemente, di potere essere utilizzata da una moltitudine di utenti di lingue e culture diverse.

La soluzione a questa esigenza è di consentire a questi utenti di poter scegliere la lingua con cui desiderano navigare le pagine dell’applicazione stessa.

L'internazionalizzazione è il processo mediante il quale vengono eliminati dal codice sorgente i preconcetti culturali quali il formato della data, la formattazione dei numeri, la visualizzazione di immagini raffiguranti elementi tipici di un determinato paese (autobus, cassette della posta, ecc.). 

La Localizzazione è il processo di adattamento di un’applicazione per permetterle di poter cambiare dinamicamente le risorse presenti nelle pagine: principalmente stringhe ma anche file di testo, immagini, file audio e ogni altro tipo di contenuto inseribile in una pagina web.
Quindi codice del tipo:

lblMessaggio.Text = “Ciao Mondo!”

deve essere riscritto in modo che la stringa “Ciao Mondo!” non sia statica, ma possa variare in modalità runtime secondo la lingua scelta dall’utente.

Django supporta pienamente la traduzione del testo, formattazione delle date, orari, numeri e  time zones.

Questo viene fatto essenzialmente attraverso due passi:
\begin{itemize}
	\item si specificano quali parti della propria app dovrebbero essere tradotte o formattate per culture e/o lingue specifiche;
	\item django usa questi "ganci" per localizzare la Web app per un utente specifico secondo le proprie impostazioni.
\end{itemize}

La parola 'Internationalization' è spesso abbreviata con 'i18n'. Questa abbreviazione è largamente usata, deriva dal fatto che ci sono 18 lettere tra la 'i' e la 'n'.

Sebbene il meccanismo di traduzione può essere avviato da molti tipi di file, per esempio nel codice python, javascript, etc... per il nostro lavoro è stato sufficiente supportare l'internazionalizzazione nell'html. Più in dettaglio, in testa alle pagine desiderate, in accordo con il linguaggio django, è stato inserito  il tag\lstinline$$, ogni elemento di cui si disporrà di una traduzione va inserito tra il tag -> \lstinline$$.

Successivamente è stato necessario scrivere la traduzione. Per far questo si utilizza il comando
\lstinline$<django-admin.py makemessages -l <id_language>>$
il quale crea un file con l'elenco delle voci da tradurre e la giusta alberatura  nel quale inserirlo. Ovviamente adesso va qui inserita la traduzione frase per frase. Successivamente lanciando il comando 
\lstinline$<django-admin.py compilemessages>$
verrà creato il file di estensione .mo utilizzato dal sistema per le traduzioni.

\endinput