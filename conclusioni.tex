\chapter{Conclusioni}
\section{Possibili sviluppi}
Ovviamente, l'applicazione ha un ampio margine di miglioramento soprattutto in termini di caratteristiche aggiuntive da offrire, e quindi da sviluppare.
Tra queste:
 \begin{itemize}
	\item specificare come indirizzo di posta elettronica quello Facebook, per ricevere novità, avvisi, o qualsiasi altra interazione;
	\item sfruttare il meccanismo delle notifiche di Facebook, ad esempio per aggiornamenti sulla nuova disponibilità di un prodotto;
	\item poter condividere una lista di oggetti desiderati attraverso un post in bacheca;
	\item sconti e/o altri vantaggi nel giorno del compleanno.
\end{itemize}

\section{Esperienza in azienda}
Valuto l'esperienza in azienda molto importante per la mia carriera in quanto mi ha messo a contatto con le reali esigenze nel mondo dello sviluppo delle applicazioni web e dell'informatica a livello extra-accademico. Il valore tecnico e la preparazione delle persone sono state un grande stimolo a dar sempre il massimo dell'impegno.

Sebbene inizialmente abbia avuto difficoltà nel capire la tecnologia, anche perchè questo è stato il primo approccio a linguaggi server-side in generale, python nello specifico, sono riuscito nel capire i concetti base grazie alle capacità di problem solving acquisite attraverso questi anni di preparazione scientifica (liceo e percorso di laurea). 

Inoltre ritengo molto utile l'esperienza in quanto mi ha permesso di valutare l'area dello sviluppo web e capire se approfondire l'argomento, magari con dei corsi di un percorso di laurea specialistica.
