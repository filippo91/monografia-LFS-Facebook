\chapter{Conclusioni}
\section{Possibili sviluppi}
Il lavoro fatto è stato pacchetizzato, ed è scaricabile da github\footnote{https://github.com/filippo91/lfs\_facebook\_buildout}. Seguendo le istruzioni inserite nel file di readme, si ricrea in pochi passi l'ambiente di sviluppo, scaricando cartelle e file del progetto con relative dipendenze in modo automatico attraverso il file di buildout.

\vspace{2.5 mm}
Ovviamente, l'applicazione ha un ampio margine di miglioramento soprattutto in termini di caratteristiche aggiuntive da offrire, e quindi da sviluppare.
Tra queste:
 \begin{itemize}
	\item specificare come indirizzo di posta elettronica quello Facebook, per ricevere novità, avvisi, o qualsiasi altra interazione;
	\item sfruttare il meccanismo delle notifiche di Facebook, ad esempio per aggiornamenti sulla nuova disponibilità di un prodotto;
	\item poter condividere una lista di oggetti desiderati attraverso un post in bacheca;
	\item sconti e/o altri vantaggi nel giorno del compleanno.
\end{itemize}

\section{Esperienza in azienda}
Valuto l'esperienza in azienda molto importante per la mia carriera in quanto mi ha messo a contatto con le reali esigenze nel mondo dello sviluppo delle applicazioni web e dell'informatica a livello extra-accademico. Il valore tecnico e la preparazione delle persone sono state un grande stimolo a dar sempre il massimo dell'impegno per apprendere il più possibile.

\vspace{2.5 mm}
Sebbene inizialmente abbia avuto difficoltà nel capire la tecnologia, anche perchè questo è stato il primo approccio a linguaggi server-side in generale, python nello specifico, sono riuscito nel capire i concetti base grazie alle capacità di problem solving acquisite attraverso questi anni di preparazione scientifica (liceo e percorso di laurea). 

\vspace{2.5 mm}
Interessante è stato esplorare il mondo open-source, leggere codice scritto da professionisti ed utilizzare questo codice per i propri progetti. Infatti, a differenza da quello che avviene spesso in ambito accademico, dove di frequente gli homework prevedono di partire da zero è realizzare tutto il codice ex novo, ho avuto l'opportunità di confrontarmi con un altro approccio allo sviluppo, ovvero quello di aggiungere funzionalità ad un progetto esistente, magari modificandone anche il comportamento nativo, ma comunque in modo da creare un plug-in, e quindi scrivendo moduli che si possano essere installati successivamente, senza modificare direttamente il codice pre-esistente. 

\vspace{2.5 mm}
Inoltre ritengo molto utile l'esperienza in quanto mi ha permesso di valutare l'area dello sviluppo web e capire se approfondire l'argomento, magari con dei corsi di un percorso di laurea specialistica.

\vspace{2.5 mm}
Infine, ho avuto modo di apprezzare la filosfia open-source la quale comporta passare da una cultura aziendale/personale protezionistica e difensiva, ad un approccio aperto, in cui tutti danno quello che sanno fare meglio e tutti si arricchiscono imparando dal meglio.

\vspace{2.5 mm}
\emph{“Se tu hai una mela, e io ho una mela, e ce le scambiamo, allora tu ed io abbiamo sempre una mela per uno. Ma se tu hai un’idea, ed io ho un’idea, e ce le scambiamo, allora abbiamo entrambi due idee.”
G.B. Shaw}