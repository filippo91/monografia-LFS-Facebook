\chapter{Conclusioni}
\section{Esperienza in azienda}
Valuto l'esperienza in azienda molto importante per la mia carriera in quanto mi ha messo a contatto con le reali esigenze nel mondo dello sviluppo delle applicazioni web e dell'informatica a livello extra-accademico. Il valore tecnico e la preparazione delle persone sono state un grande stimolo a dar sempre il massimo dell'impegno.

Sebbene inizialmente abbia avuto difficoltà nel capire la tecnologia, anche perchè questo è stato il primo approccio a linguaggi server-side in generale, python nello specifico, sono riuscito nel capire i concetti base grazie alle capacità di problem solving acquisite attraverso questi anni di preparazione scientifica (liceo e percorso di laurea). 

Inoltre ritengo molto utile l'esperienza in quanto mi ha permesso di valutare l'area dello sviluppo web e capire se approfondire l'argomento, magari con dei corsi di un percorso di laurea specialistica.

\section{Possibili sviluppi}
bozza


\begin{itemize}
	\item possibilità di mandare messaggi di fb al posto delle mail per gli utenti fb
	\item notifiche di un oggetto nuovamente disponibile via notifiche di fb e non mail
	\item wishlist condivisibile
	\item ricevere/inviare coupon via fb (magari comprare un coupon e spedirlo ad un amico)
	\item aggiungere dei metatag di facebook per aggiungere una semantica alle semplici condivisioni di prodotti. Ci sono dei metadati che possono categorizzare la pagina e non solo dire mi piace questa pagina generica, ma dire mi è piaciuto questo libro o film o qualsiasi altra cosa, con questa copertina, dell’autore X, venduto da Y
	\item Sconto nel giorno del compleanno
\end{itemize}
