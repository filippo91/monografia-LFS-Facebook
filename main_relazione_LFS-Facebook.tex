\documentclass[a4paper, twoside, 11pt]{toptesi}

\usepackage{hyperref}

\hypersetup{
    pdfpagemode={UseOutlines},
    bookmarksopen,
    pdfstartview={FitH},
    colorlinks,
    linkcolor={blue},
    citecolor={red},
    urlcolor={blue}
}

\usepackage[utf8]{inputenc}
\usepackage[T1]{fontenc}

\usepackage{listings}
\usepackage[usenames,dvipsnames]{color}

%specifiche per il frontespizio
\FacoltaDi{}
\corsodilaurea{Ingegneria Informatica}
\monografia{LFS-Facebook: integrazione di un'applicazione e-commerce sulla piattaforma Facebook}
\tutore{Marco Torchiano}
\tutoreaziendale{Fabrizio Reale}
\candidato{Filippo Projetto}

\lstdefinelanguage{JavaScript}{
  keywords={typeof, new, true, false, catch, function, return, null, catch, switch, var, if, in, while, do, else, case, break},
  keywordstyle=\color{blue}\bfseries,
  ndkeywords={class, export, boolean, throw, implements, import, this},
  ndkeywordstyle=\color{darkgray}\bfseries,
  identifierstyle=\color{black},
  sensitive=false,
  comment=[l]{//},
  morecomment=[s]{/*}{*/},
  commentstyle=\color{purple}\ttfamily,
  stringstyle=\color{red}\ttfamily,
  morestring=[b]',
  morestring=[b]"
}

%gray frame for code ;) GREAT SOLUTION!!
\usepackage{framed}
\definecolor{shadecolor}{gray}{0.95}

\begin{document}



\lstset{
	language=Python,
	basicstyle=\small, % print whole listing small
	keywordstyle=\color{black}\bfseries,% underlined bold black keywords
	identifierstyle=\color{black}, % nothing happens
	commentstyle=\color{white}, % white comments
	stringstyle=\ttfamily, % typewriter type for strings
	showstringspaces=false,
	%frame=single,
	%backgroundcolor=\color{GreenYellow},
	} 


\frontespizio

\indici

\chapter{Introduzione}

\section{L'azienda}
Dal 2004 Redomino cura lo sviluppo, la commercializzazione e la manutenzione di piattaforme digitali open source per il business e dal 2007 fornisce servizi nei settori del digital marketing, del social media marketing e più di recente del green marketing e della co-creazione. La divisione green marketing di Redomino aiuta i clienti a comunicare al meglio i propri valori e le proprie eccellenze nel settore socio-ambientale con l'obiettivo di sviluppare innovative forme di business.

\section{Obiettivi}
Il tirocinio si è incentrato sull'apprendimento del linguaggio Python, del framework Django e delle API Facebook. 
E' stata realizzata, attraverso il supporto formativo del tutor aziendale e figure specializzate all'interno dIntegrazione con Facebook Backendell'azienda, un'applicazioni web in grado di integrare come Facebook App un e-commerce. 
L'obiettivo non è stato quello di creare un sistema completo, ma dei moduli riusabili che prototipino il comportamento richiesto. 
Componenti fondamentali del tirocinio saranno anche l'interazione con le comunita' open source che seguono i suddetti progetti, e la documentazione di quanto svolto.

\section{Tecnologie al contorno}
Le tecnologie utilizzate durante lo sviluppo del progetto sono quelle ritenute molto spesso necessarie per la maggior parte dei progetti di sviluppo software. Tra queste:
\begin{itemize}
\item software di controllo versione, in particolare quello fornito da https://github.com, che è stato utile per le seguenti azioni:
	\begin{itemize}
	\item creazione di un repository: <git init>;
	\item aggiunta di file al repository: <git add “nome file”>, <git commit -m “messaggio che descrive la modifica”>, <git push>;
	\item mostra i cambiamenti tra file versionati e quelli in locale in fase di sviluppo: <git diff “nome file”>;
	\item ripristino ad una versione precedente di uno o più file del progetto, o: <git checkout “nome file”>.
	\end{itemize}
\item un build system basato su Python per creare, assemblare e sviluppare applicazioni da piu parti, alcune delle quali potrebbe non essere scritte in Python. Creare un file di configurazione Buildout permette di riprodurre lo stesso software dopo lo sviluppo dello stesso senza perdite di tempo.

“While not directly aiming to solve world peace, it perhaps will play a role in the future, as people will be less angry about application deployment and will have more time for making love and music." --Noah Gift, co-author of 'Python For Unix and Linux' from O'Reilly.”
\item virtualenv, un tool che permette di creare versioni isolate dell'ambiente Python, utile per svariati motivi, tra cui:
	\begin{itemize}
	\item ogni virtualenv ha i permessi dell’utente che lo crea, quindi per installare pacchetti addizionali non occorrono i permessi di root;
	\item lasciare l’installazione del sistema il più snella possibile;
	\item aggiornare una o più librerie di Python, senza che venga compromesso il funzionamento di altri programmi.
	\end{itemize}
\item E' stato necessario inoltre ottenere un certificato SSL, in quanto le politiche Facebook riguardanti le applicazioni prevedono come requisito una connessione sicura https. Openssl toolkit è stato utilizzato per generare una RSA Private Key e un CSR (Certificate Signing Request).  Il CSR può essere utilizzato in due modi: 
	\begin{itemize}
	\item può essere inviato ad una Certificate Authority, come Thawte or Verisign che verificherà l'identità del  richiedente ed emetterà un signed certificate;
	\item la seconda opzione è creare un CSR self-signed.
	\end{itemize}
Per i nostri scopi non è stato sufficiente ottenere un certificato self-signed.
\end{itemize}
\endinput

\chapter{LFS-Facebook}

\section{Django Framework}
Django è un framework di alto livello per Python, che ha l'obiettivo di rendere facile e veloce lo sviluppo di applicazioni web.
.... forse spiego grossolanemente cosa fanna le parti che ho modificato in django....
 
\section{Progetti di partenza}
Una fase preliminare del tirocinio è stata impiegata principalmente nello studio di due progetti open source: LFS e django-facebook.
LFS è un negozio online che  si basa su Python, Django e jQuery. A questo è stato aggiunto poi un tema responsivo, ed ha rappresentato il progetto base sul quale si è lavorato per aggiungere delle funzionalità legate alla sua integrazione come applicazione fruibile su Facebook. 

Django-facebook, invece, permette ad applicazioni django di registrare gli utenti tramite backend Facebook, convertendo i dati dell'utente in un modello consono a django.

\section{Gestione Backend}
Un'esigenza che un progetto di un'applicazione potrebbe avere è quella di dover/voler pescare da un'altra sorgente di usernames e passwords.

Questo è proprio il nostro caso, vogliamo usare come backend di autenticazione quello di Facebook.
Per iniziare a prender un po' la mano abbiamo iniziato provando a far funzionare un'applicazione django come LFS con un LDAP. 

LDAP è un protocollo di livello applicativo per accedere e gestire servizi di directory su una rete IP. Uno degli usi più comuni di LDAP è quello di fornire un “single sign-on” dove per la coppia d'informazioni utente:password è condivisa tra diversi servizi.
Nel nostro caso è stato utilizzato openLDAP, un'implementazione opensource del suddetto protocollo.


Per aggiungere questa funzionalità ad un'applicazione Django è necessario modificare la variabile AUTHENTICATION BACKENDS del file setting.py, in particolare Django mantiene una lista di “authentication backends” che egli verifica per ogni autenticazione. 

%\begin{lstlisting}[frame=single]
%	AUTHENTICATION_BACKENDS = (
%    'django_auth_ldap.backend.LDAPBackend',
%    'django.contrib.auth.backends.ModelBackend',
%	)
%\end{lstlisting}

Quando qualcuno cerca di autenticarsi, Django prova l'autenticazione lungo tutta la lista di backend per cui è impostato. 
Attenzione che l'ordine è importante in quanto django si fermerà al primo match positivo.

%Con LDAP sono necessarie una serie di altre opzioni da aggiungere al file di setting, mentre per accedere al backend di autenticazione di Facebook è sufficente aggiungere in testa alla lista AUTHENTICATION_BACKENDS 'django_facebook.auth_backends.FacebookBackend'.

\section{Integrare un applicazione come Facebook App}
Per inserire l'applicazione tra le App di Facebook è necessario procurarsi innanzitutto le credenziali, in modo da.....
Qui i passi utili per ottenerle:
\begin{itemize}
	\item raggiungere la Facebook Developers Apps page;
	\item registrarsi come sviluppatore se non lo si è già;
	\item creare una nuova applicazione cliccando sull'aposito pulsante;
	\item accettare le normative della piattaforma Facebook;
\end{itemize}
Così facendo, sulla pagina principale dell'applicazione sono disponibili le due credenziali utili per il proseguo: ID applicazione e App Secret.
Adesso vanno inserite nel file di setting del progetto:

\begin{lstlisting}[frame=single]
FACEBOOK_APP_ID = 'YOUR APP ID'
FACEBOOK_APP_SECRET = 'YOUR APP SECRET NUMBER'
\end{lstlisting}
%FACEBOOK_CANVAS_PAGE = 'https://apps.facebook.com/%s/' % FACEBOOK_APP_ID
Da notare che è richiesta una connessione https.

\section{Protezione delle viste con autenticazione richiesta}
L'applicazione si presenta senza un link ad una pagina di login, in quanto l'utente sarà automaticamente rediretto alla procedura di autenticazione quando e se necessario. 

Per ottenere questo comportamento si è scelto di creare un decoratore che protegga le viste desiderate.
Le viste da proteggere non sono imposte dal sistema, ma è tutto pienamente configurabile attravrso il file di setting. Infatti, è stato creato un dizionario di variabili booleane, a ciascuna delle quali corrisponde una pagina che è possibile proteggere settando come 'True' la variabile in questione.

\begin{lstlisting}[frame=single]
VIEW_WITH_LOGIN_REQUIRED = {
    'add-to-cart': True, #True or False
    'shop': False, #True or False
    'category': False, #True or False
    'product': False, #True or False
    'checkout': False, #True or False
}
\end{lstlisting}

Se una vista è protetta da decoratore, sarà dunque scandito l'elenco di variabili: se la variabile corrispondente è uguale a True, sarà chiamato un altro decoratore, nativo del framework Django, @login\textunderscore required, che verificherà lo stato dell'utente, che se loggato procederà verso la pagina richiesta, altrimenti sarà rediretto alla pagina di login.

\begin{lstlisting}[frame=single]
def permissions_required(view_name):
    
    def decorator(func):
        if settings.VIEW_WITH_LOGIN_REQUIRED[view_name]:
            return login_required(func)
        else:
            return func

    return decorator
\end{lstlisting}

Comunque si può scegliere di non proteggere alcuna vista e permettere la libera navigazione all'utente che agirà come utente anonimo, potendo aggiungere prodotti al carrello, andare alla cassa, decidere di completare l'acquisto e quindi effettuare il pagamento.

\subsection{Pagina di login}
La pagina di login è stata modificata facendo un override di quella esistente attraverso il linguaggio di template di django, che permette, tra le altre cose, di ereditare da una pagina html e sovrascrivere i blocchi che si vuole.

\begin{lstlisting}[language=html]
 -> inheritance
 -> blocco da sovrascrivere
...

\end{lstlisting}

richiesta di connessione in poll 

\begin{lstlisting}[language=JavaScript]

var myVar = setInterval(function(){
    if(F !== undefined){
        F.connect(document.getElementById('facebook_form'), 
        ['email', 'user_about_me', 'user_birthday', 'user_website', 'user_likes']);
        clearInterval(myVar);
    }
    }, 500);

\end{lstlisting}

\subparagraph{template override}
L'override di un template pre-esistente viene effettuato creando un file html con lo stesso identico nome e dovrà essere riprodotta l'alberatura. Per esempio se voglio sovrascrivere lfs/templates/lfs/base.html, dovrò creare un file base.html in myapp/templates/lfs/base.html. 

Dopodichè è necessario anche intervenire nel file di setting ed assicurarsi che fra le INSTALLED\_ APPS, myapp appaia prima della applicazione di cui vogliamo sovrascrivere il template.

\subsection{Pagina di checkout}
\section{Personalizzazione del modello utente}
\section{Facebook plugin e API}
\section{Prodotti riservati ai fan}
\section{Internazionalizzazione}
Lo sviluppo di un’applicazione web richiede, sempre più frequentemente, di potere essere utilizzata da una moltitudine di utenti di lingue e culture diverse.

La soluzione a questa esigenza è di consentire a questi utenti di poter scegliere la lingua con cui desiderano navigare le pagine dell’applicazione stessa.

L'internazionalizzazione è il processo mediante il quale vengono eliminati dal codice sorgente i preconcetti culturali quali il formato della data, la formattazione dei numeri, la visualizzazione di immagini raffiguranti elementi tipici di un determinato paese (autobus, cassette della posta, ecc.). 

La Localizzazione è il processo di adattamento di un’applicazione per permetterle di poter cambiare dinamicamente le risorse presenti nelle pagine: principalmente stringhe ma anche file di testo, immagini, file audio e ogni altro tipo di contenuto inseribile in una pagina web.
Quindi codice del tipo:

lblMessaggio.Text = “Ciao Mondo!”

deve essere riscritto in modo che la stringa “Ciao Mondo!” non sia statica, ma possa variare in modalità runtime secondo la lingua scelta dall’utente.

Django supporta pienamente la traduzione del testo, formattazione delle date, orari, numeri e  time zones.

Questo viene fatto essenzialmente attraverso due passi:
\begin{itemize}
	\item si specificano quali parti della propria app dovrebbero essere tradotte o formattate per culture e/o lingue specifiche;
	\item django usa questi "ganci" per localizzare la Web app per un utente specifico secondo le proprie impostazioni.
\end{itemize}
\endinput


\chapter{Conclusioni}
\section{Possibili sviluppi}
Il lavoro fatto è stato pacchetizzato, ed è scaricabile da github\footnote{https://github.com/filippo91/lfs\_facebook\_buildout}. Seguendo le istruzioni inserite nel file di readme, si ricrea in pochi passi l'ambiente di sviluppo, scaricando cartelle e file del progetto con relative dipendenze in modo automatico attraverso il file di buildout.

Ovviamente, l'applicazione ha un ampio margine di miglioramento soprattutto in termini di caratteristiche aggiuntive da offrire, e quindi da sviluppare.
Tra queste:
 \begin{itemize}
	\item specificare come indirizzo di posta elettronica quello Facebook, per ricevere novità, avvisi, o qualsiasi altra interazione;
	\item sfruttare il meccanismo delle notifiche di Facebook, ad esempio per aggiornamenti sulla nuova disponibilità di un prodotto;
	\item poter condividere una lista di oggetti desiderati attraverso un post in bacheca;
	\item sconti e/o altri vantaggi nel giorno del compleanno.
\end{itemize}

\section{Esperienza in azienda}
Valuto l'esperienza in azienda molto importante per la mia carriera in quanto mi ha messo a contatto con le reali esigenze nel mondo dello sviluppo delle applicazioni web e dell'informatica a livello extra-accademico. Il valore tecnico e la preparazione delle persone sono state un grande stimolo a dar sempre il massimo dell'impegno.

Sebbene inizialmente abbia avuto difficoltà nel capire la tecnologia, anche perchè questo è stato il primo approccio a linguaggi server-side in generale, python nello specifico, sono riuscito nel capire i concetti base grazie alle capacità di problem solving acquisite attraverso questi anni di preparazione scientifica (liceo e percorso di laurea). 

Inoltre ritengo molto utile l'esperienza in quanto mi ha permesso di valutare l'area dello sviluppo web e capire se approfondire l'argomento, magari con dei corsi di un percorso di laurea specialistica.


\end{document}